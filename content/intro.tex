\subsection{Die Akquinet GmbH}
Die 2002 gegründete Akquinet GmbH mit Sitz in Hamburg ist ein deutscher IT- Dienstleister. \cite{akquinetFirmengeschichte} Das Unternehmen bietet unter anderem Dienstleistungen in den Bereichen Softwareentwicklung, IT-Beratung und Managed Services für verschiedene Branchen wie Gesundheitswesen \cite{akquinetGesundheitswesen}, öffentlicher Sektor \cite{akquinetoffentlich}, Logistik \cite{akquinethafenlogistik} und Fertigungsindustrie \cite{akquinetindustrie}. Akquinet arbeitet mit Unternehmen wie SAP, Microsoft, IBM und Open Source und entwickelt unter anderem Lösungen für Enterprise Resource Planning (ERP), Customer Relationship Management (CRM) und Business Intelligence (BI). \newline \newline
Die Akquinet GmbH betreibt eigene hochsichere Rechenzentren in Deutschland und beschäftigt mehrere hundert Mitarbeiter.\cite{akquinetueberuns} Neben dem Hauptsitz in Hamburg gibt es weitere Standorte in Deutschland sowie Niederlassungen in Österreich und der Schweiz. Das Unternehmen legt großen Wert auf partnerschaftliche Zusammenarbeit und nachhaltige Unternehmensführung und hat zahlreiche Projekte für namhafte Kunden erfolgreich umgesetzt.\cite{akquinetFirmengeschichte}

\subsection{Kontext \& Motivation}
Die vorliegende Transferleistung widmet sich der Fragestellung, wie Unternehmen durch eine detaillierte Analyse des Freizeitverhaltens von Studierenden wertvolle Erkenntnisse für die Ableitung von Corporate Benefits (CB) gewinnen können. Eine eingehende Untersuchung und Optimierung der CB unter diesem speziellen Aspekt kann letztlich zu einer deutlich verbesserten Zielgruppenansprache und effektiveren Kommunikation mit den Studierenden führen. Unternehmen können auf Basis dieser Erkenntnisse ihre Angebote gezielter auf die Bedürfnisse und Wünsche der Studierenden abstimmen. Dies ermöglicht die Entwicklung maßgeschneiderter Benefits, die genau auf die Vorlieben und Freizeitaktivitäten der Studierenden abgestimmt sind. Ein solcher personalisierter Ansatz bei der Gestaltung von Corporate Benefits trägt dazu bei, die Zufriedenheit und das Engagement der Studierenden zu steigern und damit langfristig auch die Attraktivität des Unternehmens als Arbeitgeber zu erhöhen.

\subsection{Methoden \& Abgrenzung}
Zur Datenerhebung wird eine Befragung der Zielgruppe durchgeführt. Aus den Ergebnissen der Befragung werden dann detaillierte Empfehlungen für Corporate Benefits (CB) abgeleitet. Die Relevanz und Bedeutung von CB wird in \hyperref[sec:relevanz]{Kapitel 2.1} ausführlich erläutert und diskutiert. Die aus der Befragung gewonnenen Erkenntnisse bilden die Grundlage für die Ableitung spezifischer Benefits. Diese abgeleiteten Nutzen werden anschließend am Beispiel der Akquinet GmbH detailliert analysiert und verglichen. Der Vergleich soll aufzeigen, wie die vorgeschlagenen CB in der Praxis umgesetzt werden können und welcher Nutzen sich daraus für die Zielgruppe ergibt. Dabei wird besonderer Wert darauf gelegt, praxisrelevante und umsetzbare Empfehlungen zu formulieren, die sowohl den Bedürfnissen der Studierenden als auch den strategischen Zielen der Akquinet GmbH gerecht werden. Damit wird aufgezeigt, wie theoretische Erkenntnisse in konkrete Maßnahmen umgesetzt werden können.