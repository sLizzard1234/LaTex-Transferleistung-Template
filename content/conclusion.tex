Die vorliegende Arbeit analysiert die Akquinet GmbH und ihre CB umfassend, wobei ein besonderes Augenmerk auf die Bedürfnisse und Präferenzen der Studierenden als Zielgruppe gelegt wird. Die detaillierte Untersuchung der vorhandenen Benefits und deren Relevanz für die Mitarbeiter zeigt, dass die Akquinet GmbH bereits eine solide Basis an CB etabliert hat, welche die Zufriedenheit und Loyalität der Mitarbeitenden fördert. Hervorzuheben sind insbesondere die Möglichkeiten zur Weiterbildung, die flexible Gestaltung der Arbeitszeiten, die Bezuschussung des öffentlichen Personennahverkehrs sowie das Angebot an Betriebssport.\newline \newline
Die empirische Umfrage hat weitere wertvolle Erkenntnisse geliefert, die zur Optimierung des bestehenden Benefit-Programms beitragen können. Die Resultate legen nahe, dass Weiterbildung und Freizeitaktivitäten für die befragte Zielgruppe von hoher Relevanz sind. Die Akquinet GmbH hat die Möglichkeit, durch die Integration dieser Präferenzen in ihr CB-Angebot ihre Attraktivität als Arbeitgeber zu steigern. Insbesondere durch die Einführung spezifischer Programme, die auf die identifizierten Freizeitaktivitäten der Studierenden zugeschnitten sind, könnte die Mitarbeiterzufriedenheit und -bindung langfristig erhöht werden.\newline \newline
Insgesamt zeigt die Arbeit, dass die Akquinet GmbH bereits über ein gut durchdachtes und effektives Benefits-Programm verfügt, das jedoch durch gezielte Maßnahmen und regelmäßige Evaluationen noch weiter verbessert werden kann. Dies würde nicht nur die Motivation und Produktivität der aktuellen Mitarbeitenden fördern, sondern auch die Rekrutierung neuer Talente erleichtern, wodurch die Wettbewerbsfähigkeit des Unternehmens nachhaltig gestärkt wird.